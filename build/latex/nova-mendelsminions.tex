%% Generated by Sphinx.
\def\sphinxdocclass{report}
\documentclass[letterpaper,10pt,english,openany,oneside]{sphinxmanual}
\ifdefined\pdfpxdimen
   \let\sphinxpxdimen\pdfpxdimen\else\newdimen\sphinxpxdimen
\fi \sphinxpxdimen=.75bp\relax

\PassOptionsToPackage{warn}{textcomp}
\usepackage[utf8]{inputenc}
\ifdefined\DeclareUnicodeCharacter
% support both utf8 and utf8x syntaxes
  \ifdefined\DeclareUnicodeCharacterAsOptional
    \def\sphinxDUC#1{\DeclareUnicodeCharacter{"#1}}
  \else
    \let\sphinxDUC\DeclareUnicodeCharacter
  \fi
  \sphinxDUC{00A0}{\nobreakspace}
  \sphinxDUC{2500}{\sphinxunichar{2500}}
  \sphinxDUC{2502}{\sphinxunichar{2502}}
  \sphinxDUC{2514}{\sphinxunichar{2514}}
  \sphinxDUC{251C}{\sphinxunichar{251C}}
  \sphinxDUC{2572}{\textbackslash}
\fi
\usepackage{cmap}
\usepackage[T1]{fontenc}
\usepackage{amsmath,amssymb,amstext}
\usepackage{babel}



\usepackage{times}
\expandafter\ifx\csname T@LGR\endcsname\relax
\else
% LGR was declared as font encoding
  \substitutefont{LGR}{\rmdefault}{cmr}
  \substitutefont{LGR}{\sfdefault}{cmss}
  \substitutefont{LGR}{\ttdefault}{cmtt}
\fi
\expandafter\ifx\csname T@X2\endcsname\relax
  \expandafter\ifx\csname T@T2A\endcsname\relax
  \else
  % T2A was declared as font encoding
    \substitutefont{T2A}{\rmdefault}{cmr}
    \substitutefont{T2A}{\sfdefault}{cmss}
    \substitutefont{T2A}{\ttdefault}{cmtt}
  \fi
\else
% X2 was declared as font encoding
  \substitutefont{X2}{\rmdefault}{cmr}
  \substitutefont{X2}{\sfdefault}{cmss}
  \substitutefont{X2}{\ttdefault}{cmtt}
\fi


\usepackage[Bjarne]{fncychap}
\usepackage{sphinx}

\fvset{fontsize=\small}
\usepackage{geometry}


% Include hyperref last.
\usepackage{hyperref}
% Fix anchor placement for figures with captions.
\usepackage{hypcap}% it must be loaded after hyperref.
% Set up styles of URL: it should be placed after hyperref.
\urlstyle{same}
\addto\captionsenglish{\renewcommand{\contentsname}{Contents:}}

\usepackage{sphinxmessages}
\setcounter{tocdepth}{1}



\title{Nova \sphinxhyphen{} Mendel\textquotesingle{}s Minions}
\date{Dec 27, 2020}
\release{}
\author{Vincent Meunier}
\newcommand{\sphinxlogo}{\vbox{}}
\renewcommand{\releasename}{}
\makeindex
\begin{document}

\pagestyle{empty}
\sphinxmaketitle
\pagestyle{plain}
\sphinxtableofcontents
\pagestyle{normal}
\phantomsection\label{\detokenize{index::doc}}



\chapter{Welcome to \sphinxstyleemphasis{Mendel’s Minions}}
\label{\detokenize{introduction:welcome-to-mendel-s-minions}}\label{\detokenize{introduction:introduction}}\label{\detokenize{introduction::doc}}
This Nova award is designed to help you explore how genetic information affects your life every day.

\begin{figure}[htbp]
\centering
\capstart

\noindent\sphinxincludegraphics[width=400\sphinxpxdimen]{{dna}.jpg}
\caption{An artist’s view of DNA, image obtained from \sphinxhref{https://www.newscientist.com/term/dna/}{NewScientist}.}\label{\detokenize{introduction:id1}}\end{figure}

\begin{sphinxadmonition}{warning}{Warning:}
When completing this Award both the youth and involved adult leaders must obey all rules of \sphinxhref{https://www.scouting.org/health-and-safety/gss/}{Safe Scouting}. This includes (1) Completing Cyber Chip training prior to starting this activity and (2) \sphinxstylestrong{ALWAYS} involve at least 2 adults in all your communications with a leader, including online. If you send an email to your counselor, always add the address of another adult leader or a parent/guardian. Never reply to a message sent by an adult leader unless another adult has been copied on the email. Report any issue to your parents/guardians!
\end{sphinxadmonition}


\section{Instructions}
\label{\detokenize{introduction:instructions}}\begin{enumerate}
\sphinxsetlistlabels{\arabic}{enumi}{enumii}{}{.}%
\item {} 
Identify a \sphinxstylestrong{Nova Counselor} either within your unit, district, or council.

\item {} 
This site provides you a platform for learning and you can easily follow all requirements using the navigation menu on the left.

\item {} 
Once you have identified a Counselor, you can start working on requirements.

\item {} 
The most important aspect in any scientific endeavor is to \sphinxstylestrong{properly document progress}. This will be done, here, using a google sheet as described in more details below.

\end{enumerate}


\section{Documenting your progress}
\label{\detokenize{introduction:documenting-your-progress}}\begin{enumerate}
\sphinxsetlistlabels{\arabic}{enumi}{enumii}{}{.}%
\item {} 
A template worksheet can be found \sphinxhref{https://docs.google.com/document/d/1Hoqz-rU-vgZ\_VLSfCU9onEyMMCR3jnbiL0DdHXuHA-Y/edit?usp=sharing}{here}. This is a \sphinxstyleemphasis{Google document}. \sphinxstylestrong{You will not be able to modify it until you make your own copy as I will now describe for you.}

\item {} 
Once you have opened the file on google doc, go to \sphinxcode{\sphinxupquote{File}} \(\rightarrow\) \sphinxcode{\sphinxupquote{Make a Copy}}.

\item {} 
Save the file with the following name: \sphinxstyleemphasis{Nova\_designed\_to\_crunch\_FIRSTNAME\_LASTNAME}

\item {} 
You will use that file to enter your progress and share with your counselor.

\item {} \begin{description}
\item[{You can share your own copy of the worksheet with your counselor using the following procedure.}] \leavevmode\begin{enumerate}
\sphinxsetlistlabels{\alph}{enumii}{enumiii}{}{)}%
\item {} 
Click on the SHARE button on the top\sphinxhyphen{}right.

\item {} 
Click on “get link”.

\item {} 
Send the link to your counselor.

\end{enumerate}

\end{description}

\end{enumerate}

\begin{sphinxadmonition}{note}{Note:}
This document provides you a guide to complete the Nova award! All requirements are marked with the following symbol: \(\boxed{\mathbb{REQ}\Large \rightsquigarrow}\). In addition, a number of fun \sphinxstyleemphasis{Additional Challenges} are provided in boxes for your entertainment.
\end{sphinxadmonition}


\section{If you have any question}
\label{\detokenize{introduction:if-you-have-any-question}}
Contact your counselor or your scoutmaster! If you have questions about the program, contact Vincent Meunier  by \sphinxhref{mailto:vinmeunier@gmail.com}{email} (as usual, make sure you copy an additional adult to all your communications with a leader!).


\section{Other Nova modules in this series}
\label{\detokenize{introduction:other-nova-modules-in-this-series}}
\begin{sphinxadmonition}{note}{Science}


\begin{savenotes}\sphinxattablestart
\centering
\begin{tabulary}{\linewidth}[t]{|T|}
\hline

\sphinxhref{https://novashoot.readthedocs.io}{\sphinxincludegraphics[scale=0.65]{{logo-shoot_black}.png}}

\sphinxhref{https://novalig.readthedocs.io}{\sphinxincludegraphics[scale=0.65]{{logo-lig_black}.png}}

\sphinxhref{https://novasplash.readthedocs.io}{\sphinxincludegraphics[scale=0.65]{{logo-splash_black}.png}}

\sphinxhref{https://novamendel.readthedocs.io}{\sphinxincludegraphics[scale=0.65]{{logo-minions_B}.png}}
\\
\hline
\end{tabulary}
\par
\sphinxattableend\end{savenotes}
\end{sphinxadmonition}

\begin{sphinxadmonition}{note}{Technology}


\begin{savenotes}\sphinxattablestart
\centering
\begin{tabulary}{\linewidth}[t]{|T|}
\hline

\sphinxhref{https://novamendel.readthedocs.io}{\sphinxincludegraphics[scale=0.65]{{logo-engines_B}.png}}

\sphinxhref{https://novaworld.readthedocs.io}{\sphinxincludegraphics[scale=0.65]{{logo-world_B}.png}}
\\
\hline
\end{tabulary}
\par
\sphinxattableend\end{savenotes}
\end{sphinxadmonition}

\begin{sphinxadmonition}{note}{Engineering}


\begin{savenotes}\sphinxattablestart
\centering
\begin{tabulary}{\linewidth}[t]{|T|}
\hline

\sphinxhref{https://novawhoosh.readthedocs.io}{\sphinxincludegraphics[scale=0.65]{{logo-whoosh_B}.png}}

\sphinxhref{https://novaupandaway.readthedocs.io}{\sphinxincludegraphics[scale=0.65]{{logo-upandaway_B}.png}}

\sphinxhref{https://novanext.readthedocs.io}{\sphinxincludegraphics[scale=0.65]{{logo-next_B}.png}}
\\
\hline
\end{tabulary}
\par
\sphinxattableend\end{savenotes}
\end{sphinxadmonition}

\begin{sphinxadmonition}{note}{Math}


\begin{savenotes}\sphinxattablestart
\centering
\begin{tabulary}{\linewidth}[t]{|T|}
\hline

\sphinxhref{https://novadtc.readthedocs.io}{\sphinxincludegraphics[scale=0.65]{{logo-dtc2_black}.png}}
\\
\hline
\end{tabulary}
\par
\sphinxattableend\end{savenotes}
\end{sphinxadmonition}


\chapter{Requirement \#1: Research and Reading}
\label{\detokenize{requirement1:requirement-1-research-and-reading}}\label{\detokenize{requirement1::doc}}
\(\boxed{\mathbb{REQ}\Large \rightsquigarrow}\) Choose A or B or C and complete ALL the requirements.
\begin{enumerate}
\sphinxsetlistlabels{\Alph}{enumi}{enumii}{}{.}%
\item {} 
Watch not less than three hours total of shows or documentaries that discuss genetics and/or genomics. Then do the following:
\begin{enumerate}
\sphinxsetlistlabels{\arabic}{enumii}{enumiii}{(}{)}%
\item {} 
Make a list of at least five questions or ideas from the show(s) you watched.

\item {} 
Discuss two of the questions or ideas with your counselor.

\end{enumerate}

\begin{sphinxadmonition}{tip}{Tip:}
Some examples include—but are not limited to—shows found on PBS (NOVA: “Cracking the Code”), Discovery Channel, Science Channel, National Geographic Channel, and TED Talks (online videos). You may choose to watch a live performance or movie developed by a local museum or state or federal agency. You may watch online productions with your counselor’s approval and under your parent’s supervision.
\end{sphinxadmonition}

\item {} 
Read (not less than three hours total) about genetics or genomics. Then do the following:
\begin{enumerate}
\sphinxsetlistlabels{\arabic}{enumii}{enumiii}{(}{)}%
\item {} 
Make a list of at least five questions or ideas from each article.

\item {} 
Discuss two of the questions or ideas with your counselor.

\end{enumerate}

\begin{sphinxadmonition}{tip}{Tip:}
Examples of magazines include—but are not limited to—Odyssey, Popular Science, Science Illustrated, Natural History, Scientific American, Nature Conservancy, Sage, and Smithsonian. Books might include The Immortal Life of Henrietta Lacks or The Gene: An Intimate History.
\end{sphinxadmonition}

\item {} 
Do a combination of reading and watching (not less than three hours total). Then do the following:
\begin{enumerate}
\sphinxsetlistlabels{\arabic}{enumii}{enumiii}{(}{)}%
\item {} 
Make a list of at least two questions or ideas from each article or show.

\item {} 
Discuss two of the questions or ideas with your counselor.

\end{enumerate}

\end{enumerate}

\begin{sphinxadmonition}{note}{Note:}
DNA is nature’s way to store genetic information. \sphinxstyleemphasis{Do you know what DNA stands for?}
Answer: \sphinxstyleemphasis{deoxyribonucleic acid}

Other important molecules in genetics are also known by their short names, like \sphinxstyleemphasis{RNA}, which stands for \sphinxstyleemphasis{ribonucleic acid}. RNA is used in your body to \sphinxstyleemphasis{carry} genetic information.

\begin{figure}[H]
\centering
\capstart

\noindent\sphinxincludegraphics[width=400\sphinxpxdimen]{{dna}.jpg}
\caption{An artist’s view of DNA, image obtained from \sphinxhref{https://www.newscientist.com/term/dna/}{NewScientist}.}\label{\detokenize{requirement1:id1}}\end{figure}
\end{sphinxadmonition}

\begin{sphinxadmonition}{attention}{Attention:}
Once you have completed this requirement, make sure you document it in your worksheet!
\end{sphinxadmonition}


\chapter{Requirement \#2: Merit Badge}
\label{\detokenize{requirement2:requirement-2-merit-badge}}\label{\detokenize{requirement2::doc}}
\(\boxed{\mathbb{REQ}\Large \rightsquigarrow}\) Complete ONE merit badge from the following list. Choose one that you have not already used toward another Nova award.
After completion, discuss with your counselor the genetic component of the merit badge you selected.
\begin{itemize}
\item {} 
Animal Science

\item {} 
Bird Study

\item {} 
Forestry

\item {} 
Gardening

\item {} 
Insect Study

\item {} 
Mammal Study

\item {} 
Medicine

\item {} 
Nature

\item {} 
Public Health

\item {} 
Reptile and Amphibian Study

\item {} 
Veterinary Medicine

\end{itemize}

\begin{figure}[htbp]
\centering

\noindent\sphinxincludegraphics[width=700\sphinxpxdimen]{{meritbadges}.png}
\end{figure}

\begin{sphinxadmonition}{attention}{Attention:}
Once you have completed this requirement, make sure you document it in your worksheet!
\end{sphinxadmonition}


\chapter{Requirement \#3: Hands\sphinxhyphen{}on activities}
\label{\detokenize{requirement3:requirement-3-hands-on-activities}}\label{\detokenize{requirement3::doc}}
\(\boxed{\mathbb{REQ}\Large \rightsquigarrow}\) Complete two of the following activities:
\begin{enumerate}
\sphinxsetlistlabels{\Alph}{enumi}{enumii}{}{.}%
\item {} 
Teach the basics of genetic inheritance to your patrol (or similar group), using gummy bear genetics or a similar method.

\begin{sphinxadmonition}{tip}{Tip:}
Helpful Link: \sphinxhref{https://my.nsta.org/resource/5243/making-mendels-model-manageable}{Making Mendel’s Model Manageable}
\end{sphinxadmonition}

\item {} 
Extract DNA from saliva, strawberries, or a banana.

\item {} 
Grow at least three generations of pea plants and explain the inheritance patterns.

\begin{sphinxadmonition}{tip}{Tip:}
Helpful Link: \sphinxhref{http://science.lovetoknow.com/life-sciences/gregor-mendels-pea-plant-experiment}{Gregor Mendel’s Pea Plant Experiment}
\end{sphinxadmonition}

\item {} 
Create a three\sphinxhyphen{}dimensional model of DNA and explain how it leads to the production of proteins.

\end{enumerate}

\begin{figure}[htbp]
\centering
\capstart

\noindent\sphinxincludegraphics[width=400\sphinxpxdimen]{{babyeyecolor.jpg}.webp}
\caption{Example on how genetics determines a baby eyes’ color. Image obtained from the \sphinxurl{https://www.momjunction.com/baby-eye-color-calculator/}  website.}\label{\detokenize{requirement3:id1}}\end{figure}

\begin{sphinxadmonition}{attention}{Attention:}
Once you have completed this requirement, make sure you document it in your worksheet!
\end{sphinxadmonition}


\chapter{Requirement \#4: Research}
\label{\detokenize{requirement4:requirement-4-research}}\label{\detokenize{requirement4::doc}}
\begin{figure}[htbp]
\centering
\capstart

\noindent\sphinxincludegraphics[width=500\sphinxpxdimen]{{dnalife}.jpg}
\caption{Image describing the Human Genome Project. This image was obtained from \sphinxurl{https://novaonline.nvcc.edu/eli/evans/his135/events/genome00/genome00.html}}\label{\detokenize{requirement4:id1}}\end{figure}

\(\boxed{\mathbb{REQ}\Large \rightsquigarrow}\) Present a report of at least 800 words or 10 minutes (with visual aids) on one of the following opics. Make sure to include the ethical issues involved in your topic. If possible, present our report to your unit or another group in addition to presenting to your counselor.
\begin{enumerate}
\sphinxsetlistlabels{\Alph}{enumi}{enumii}{}{.}%
\item {} 
Mendelian inheritance, DNA, RNA, genetics, genomics, sequencing, and Punnett squares

\item {} 
Genetic diseases, personalized medicine, and genetic counseling

\item {} 
Genetically modified food, transgenic animals, and hybrid foods

\item {} 
Use of large genetics databases for forensic analysis/solving crimes, genealogical research, or medical studies (The Cancer Genome Atlas (TCGA), Catalogue of Somatic Mutations in Cancer (COSMIC), Exome Aggregation Consortioum (ExAC), 100,000 Genomes Project, mitochondria! DNA, etc.)

\item {} 
Pharmacogenetics and oncogenomics

\item {} 
Human Genome Project

\item {} 
CRISPR (Clustered Regularly\sphinxhyphen{}Interspaced Short Palindromic Repeats)

\item {} 
Biotechnology, biologics, and biosimilar drugs

\item {} 
Another related topic approved by your counselor in advance

\end{enumerate}

\begin{sphinxadmonition}{note}{Additional Challenge}

\sphinxstyleemphasis{Britannica kids} has an excellent introduction to \sphinxstylestrong{genetic engineering} website you can find \sphinxhref{https://kids.britannica.com/kids/article/genetic-engineering/600760}{here} where you can learn more about how it may be used to create helpful medical substances, such as vaccines.
\end{sphinxadmonition}

\begin{sphinxadmonition}{note}{Note:}\begin{description}
\item[{CRISPR}] \leavevmode
Did you know that Nobel Prize in Chemistry in 2020 was awarded to Dr. Emmanuelle Charpentier (from the Max Planck Unit for the Science of Pathogens, Berlin, Germany) and to Dr. Jennifer A. Doudna (from University of California, Berkeley, USA) \sphinxstyleemphasis{for the development of a method for genome editing}, more specifically for developping CRISPR. Read more about it on the Nobel Prize \sphinxhref{https://www.nobelprize.org/prizes/chemistry/2020/press-release/}{website}.

\end{description}

\begin{figure}[H]
\centering
\capstart

\noindent\sphinxincludegraphics[width=400\sphinxpxdimen]{{nobelchemistry2020}.jpg}
\caption{CRISPR scientists Jennifer Doudna (left) and Emmanuelle Charpentier have won 2020  Nobel Prize in chemistry (image from \sphinxhref{https://www.statnews.com/2020/10/07/two-crispr-scientists-win-nobel-prize-in-chemistry/}{statnews.com})}\label{\detokenize{requirement4:id2}}\end{figure}
\end{sphinxadmonition}

\begin{sphinxadmonition}{attention}{Attention:}
Once you have completed this requirement, make sure you document it in your worksheet!
\end{sphinxadmonition}


\chapter{Requirement \#5: Visit}
\label{\detokenize{requirement5:requirement-5-visit}}\label{\detokenize{requirement5::doc}}
\(\boxed{\mathbb{REQ}\Large \rightsquigarrow}\) Visit a place where genetic and genomic information is being used, e.g., a biotechnology company or research lab, genetic counselor, a physician’s office, crime lab, zoo, natural history museum, farm, hatchery, nursery/greenhouse, etc. Discuss the work done there with one (or more) of the employees. Find out how they are using genetic information, how it has changed their work, and what they look forward to in the future.

\begin{sphinxadmonition}{note}{Additional Challenge}

You have certainly heard about GMO (or \sphinxstyleemphasis{genetically modified organism}). A GMO is a laboratory process of taking genes from one species and inserting them into another in an attempt to obtain a desired trait or characteristic.

You can learn more about GMOs on a number of website, such as this \sphinxhref{https://study.com/academy/lesson/gmos-lesson-for-kids.html}{one} from study.com or this \sphinxhref{https://www.livescience.com/40895-gmo-facts.html}{one} from livescience.com. You can learn more about their use but also about controversy on their effects of health. Ask your counselor for more information if you want to learn beyond the infomation provided in these two websites!
\end{sphinxadmonition}

\begin{sphinxadmonition}{tip}{Tip:}
A chromosome is a strand of DNA that is encoded with genes. In most cells, humans have 22 pairs of these chromosomes plus the two sex chromosomes (XX in females and XY in males) for a total of 46. (definition from www.vocabulary.com).

\begin{figure}[H]
\centering
\capstart

\noindent\sphinxincludegraphics[width=500\sphinxpxdimen]{{chromosome}.jpg}
\caption{An artist’s view of a chromosome. (Image obtained from \sphinxurl{https://online.stanford.edu/courses/xgen101-fundamentals-genetics-genetics-you-need-know})}\label{\detokenize{requirement5:id1}}\end{figure}
\end{sphinxadmonition}

\begin{sphinxadmonition}{attention}{Attention:}
Once you have completed this requirement, make sure you document it in your worksheet!
\end{sphinxadmonition}


\chapter{Requirement \#6: Genetics @ life}
\label{\detokenize{requirement6:requirement-6-genetics-life}}\label{\detokenize{requirement6::doc}}
\(\boxed{\mathbb{REQ}\Large \rightsquigarrow}\) Discuss with your counselor how genetics and genomics affect your everyday life and how you think it will affect your life in the future.

\begin{sphinxadmonition}{note}{Additional Challenge}

Information is coded in DNA with 4 small molecules, also known as \sphinxstyleemphasis{bases}. Those molecules are usually referred to a “A”, “C”, “G”, and “T”. Can you find out what those letters stand for? As an additional challenge, find out what molecules are used to code information in RNA (see definition under Requirement \#1).
\end{sphinxadmonition}

\begin{sphinxadmonition}{tip}{Tip:}
Genetic testing is a type of medical test that identifies changes in chromosomes, genes, or proteins. The results of a genetic test can confirm or rule out a suspected genetic condition or help determine a person’s chance of developing or passing on a genetic disorder.

In addition to being used as medical tool, a number of people now test their DNA for tracing their roots and finding relatives.

\begin{figure}[H]
\centering
\capstart

\noindent\sphinxincludegraphics[width=300\sphinxpxdimen]{{DNA-cancer-genetics}.jpg}
\caption{Genetic testing can be done to find out about inherited mytations ties to a disease such as cancer, and thus take precautions before a cancer develops.  (Image obtained from \sphinxurl{https://www.uclahealth.org/cancer-genetics/genetic-testing-for-cancer} where a short description is provided to show how genetic testing for cancer works).}\label{\detokenize{requirement6:id1}}\end{figure}
\end{sphinxadmonition}

\begin{sphinxadmonition}{tip}{Tip:}
Additional resources
\begin{itemize}
\item {} 
National Institute of General Medical Sciences: \sphinxhref{www.nigms.nih.gov/Pages/default.aspx}{here}

\item {} 
Your Genome—Organisms That Have Had Their Genomes Sequenced: \sphinxhref{www.yourgenome.org/facts/timeline-organisms-that-have-had-their-genomes-sequenced}{here}

\end{itemize}
\end{sphinxadmonition}

\begin{sphinxadmonition}{attention}{Attention:}
Once you have completed this requirement, make sure you document it in your worksheet!
\end{sphinxadmonition}


\chapter{About the author}
\label{\detokenize{contact:about-the-author}}\label{\detokenize{contact::doc}}
These pages were written by Vincent Meunier, the Chair of the STEM committee of \sphinxhref{https://www.trcscouting.org}{Twin Rivers Council} in New York State.

Vincent Meunier is a Professor of physics at Rensselaer Polytechnic Institute. If you have any questions, feel free to contact him by \sphinxhref{mailto:vinmeunier@gmail.com}{email}.

\begin{sphinxadmonition}{note}{Note:}
Most of the material used here was obtained from a number of external scouting sources, including \sphinxhref{https://www.scouting.org/wp-content/uploads/2018/11/Designed-to-Crunch-Nova-2018Nov26.pdf}{scouting.org}
\end{sphinxadmonition}

\begin{sphinxadmonition}{note}{Note:}
Gregor Johann Mendel (1822 \textendash{} 1884) was a scientist, meteorologist, mathematician, biologist, Augustinian friar and abbot of St. Thomas’ Abbey in Brno, Margraviate of Moravia. He gained posthumous recognition as the founder of the modern science of genetics. Though farmers had known for millennia that crossbreeding of animals and plants could favor certain desirable traits, Mendel’s pea plant experiments conducted between 1856 and 1863 established many of the rules of heredity, now referred to as the laws of Mendelian inheritance. (text adapted from \sphinxhref{https://en.wikipedia.org/wiki/Gregor\_Mendel}{wikipedia}).

\begin{figure}[H]
\centering
\capstart

\noindent\sphinxincludegraphics[width=400\sphinxpxdimen]{{gregor-mendel}.jpg}
\caption{Gregor Mendel making observations in peas (Image obtained from genoma.com)}\label{\detokenize{index:id1}}\end{figure}
\end{sphinxadmonition}



\renewcommand{\indexname}{Index}
\printindex
\end{document}